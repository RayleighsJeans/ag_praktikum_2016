\documentclass[a4paper,12pt,titlepage,fleqn]{article}%,twoside
\input{Laborbuch.pre}
\usepackage{hyperref}%[bookmarksopen]
\usepackage{ifthen}
\hypersetup{%
bookmarksnumbered=true,
pdftitle = {Photodetachment},
pdfsubject = {Laborbuch},
pdfauthor = {C112 Crew},
pdfkeywords = {},
pdfcreator = {pdflatex},
pdfproducer = {pdflatex},
linktocpage = {true}
}
\hyphenation{Pump-strahls New-port Spec-tra}

\usepackage{mathtools}

\usepackage{color}
\usepackage{tocloft} % keine POC-Puenktchen
\renewcommand{\cftdotsep}{\cftnodots}
\let\thepage=\relax % keine TOC-Seitenzahlen

\usepackage{setspace}
\usepackage{placeins}
\usepackage{textcomp}

\newenvironment{Liste}[1][$\bullet$] % Liste mit gro�em linken Abstand
{\begin{list}{#1}{\setlength{\labelwidth}{2cm}\setlength{\labelsep}{0.5cm}\setlength{\itemindent}{0cm}\setlength{\leftmargin}{2.5cm}}\setlength{\itemsep}{0cm}}
{\end{list}}	
\newenvironment{Liste2}[1][$\bullet$] % Liste mit kleinem linken Abstand
{\begin{list}{#1}{\setlength{\labelwidth}{0.5cm}\setlength{\labelsep}{0.5cm}\setlength{\itemindent}{0cm}\setlength{\leftmargin}{1.5cm}}}
{\end{list}}	

% Gleichungsnummerierung
\newcounter{EqnNr}
\renewcommand{\theequation}{\thesubsection.\arabic{EqnNr}
}
%\newcommand{\refeq}[1]{(\ref{#1})}

\newcounter{nummer}
\newenvironment{Nummer}[1][(\arabic{nummer})]
{\begin{list}{#1}{\usecounter{nummer}\setlength{\labelwidth}{1cm}\setlength{\labelsep}{0.25cm}\setlength{\itemindent}{0cm}\setlength{\leftmargin}{1cm}}}
{\end{list}}

\usepackage{tikz}
\newcommand{\Bild}[2][]{\begin{center}%\begin{figure}[htbp]
\includegraphics[#1]{#2}
\end{center}
}%\end{figure}

\newcommand{\Bildref}[3][]{(Abb. \ref{#2})\begin{figure}[htb]\centering
\includegraphics[#1]{#2}
	\caption{\label{#2}#3}
\end{figure}}

\newcommand{\Bildquer}[2][]{
	\addtolength{\textwidth}{2cm}
	\addtolength{\leftmargin}{-1cm}
		\fancyfoot[C]{}
		\begin{landscape}\centering
			\includegraphics[#1]{#2}
		\end{landscape}
	\addtolength{\textwidth}{-2cm}
	\addtolength{\leftmargin}{1cm}
	\fancyfoot[C]{\thesubsection}
}


\newcommand{\Rot}[1]{{\color[rgb]{1,0,0}#1}}%
\newcommand{\blau}[1]{{\color[rgb]{0,0,1}#1}}%
\newcommand{\gruen}[1]{{\color[rgb]{0,0.5,0}#1}}%
\newcommand{\orange}[1]{{ \color[rgb]{0.8,0.4,0}#1}}%
\newcommand{\rosa}[1]{{\color[rgb]{0.8,0,0.8}#1}}%
\newcommand{\weiss}[1]{{\color[rgb]{1,1,1}#1}}%
\newcommand{\braun}[1]{{\color[rgb]{0.5,0.25,0}#1}}%

\definecolor{darkgreen}{rgb}{0,0.5,0}
\definecolor{lila}{rgb}{0.8,0.2,1}
\definecolor{orange}{rgb}{1,0.5,0}
\definecolor{braun}{rgb}{0.5,0.25,0}

% Listenkommandos
\newcommand{\itemto}{\item[\textsf{\blau{$\boldsymbol\to$}}]}
\newcommand{\itemachtung}{\item[\textsf{\colorbox{red}{\weiss{!!!}}}]}
\newcommand{\itemfazit}{\item[\textsf{\colorbox{darkgreen}{\weiss{ $\boldsymbol\Rightarrow$}}}]}
\newcommand{\itemwas}{\FloatBarrier\item[\textsf{\colorbox{lila}{\weiss{??}}}]}
\newcommand{\itemok}{\item[\textsf{\gruen{$\boldsymbol\checkmark$}}]}
\newcommand{\itemursache}{\item[\textsf{\colorbox{blue}{\weiss{?!}}}]}
\newcommand{\itemfrage}{\item[\textsf{\colorbox{orange}{\weiss{?}}}]}
\newcommand{\itemPar}[1]{\item[\textsf{\fcolorbox{braun}{white}{\braun{#1}}}]}
\newcommand{\itemNummer}[1]{\item[\textsf{\fcolorbox{red}{white}{\Rot{#1}}}]}
\newcommand{\leer}{\vspace{2ex}}

% Monate und Tage
\newcommand{\neuermonat}[1]{\clearpage\newpage
\addtocontents{toc}{\newpage{\LARGE\bf #1}\newline}
\addtocontents{toc}{{\LARGE\bf\hrule}}
{\centering\section{ #1}\thispagestyle{empty}}}

\newcommand{\neuertag}[5][nix]{
\setcounter{EqnNr}{0}
\clearpage\newpage\thispagestyle{empty}\setcounter{jahr}{#2}\setcounter{monat}{#3}\setcounter{tag}{#4}\ifthenelse{\equal{#1}{nix}}{\subsection{#5}}{\subsection[#1]{#5}}\suppressfloats[t]}

\renewcommand{\thesection}{}
\renewcommand{\thesubsection}{
\arabic{jahr}-\arabic{monat}-\arabic{tag}}
\renewcommand{\thefigure}{\Roman{figure}}

% Einstellungen f�r Gleitobjekte
\renewcommand{\topfraction}{0.9}
\renewcommand{\textfraction}{0.01}

% andere Schriftart f�r Dateinamen
\newcommand{\Datei}[1]{{\tt #1}}

\newcounter{jahr}
\newcounter{monat}
\newcounter{tag}

\makeatletter \@addtoreset{figure}{subsection} \makeatother

% Kopf und Fusszeilen
%\usepackage[font=footnotesize,labelfont=bf]{caption}

%\usepackage{caption}[2003/12/20]
\usepackage{fancyhdr}
\pagestyle{fancy}
% Oder kurz: \fancyheader[EL,OR]{\thepage}}]
\fancyhead[L]{\leftmark}% Kapitel/Abschnitt
\fancyhead[R]{\rightmark}% Abschnitt/Unterabschnitt

\fancyhf{}
\fancyhead[C]{\rightmark}
\fancyfoot[C]{\thesubsection}

\setlength{\headheight}{16pt}
\typeout{hier:\the\topskip}
\setlength{\textheight}{25cm}
\setlength{\topmargin}{-2cm}

\fancyfoot[C]{\thesubsection}


\setlength{\mathindent}{0pt} % Einzug bei Formeln

\usepackage{txfonts}

\usepackage{forloop}
\newcounter{ct}
\usepackage{pdflscape}

\usepackage{grffile}% for file names with .

\begin{document}
\renewcommand{\figurename}{Abb.} % nur Abk�rzungen in Bildunterschrift
\renewcommand{\tablename}{Tab.}

\title{Laboratory book C112}

\author{C112 Crew}
%\institute{Institut f�r Physik\\Ernst-Moritz-Arndt-Universit�t Greifswald}
\begin{titlepage}
\maketitle

\end{titlepage}
\newpage
\thispagestyle{empty}
\cleardoublepage


\pagestyle{empty}
\tableofcontents

\pagestyle{fancy}
\fancyfoot[C]{\thesubsection}
\setlength{\mathindent}{0pt}

\addtocontents{toc}{\protect\setlength{\cftsubsecnumwidth}{2.5cm}}

\neuermonat{June 2016}

\neuertag{2016}{06}{15}{Axial of line ratio wavelength}
\begin{Liste}
	\itemwas ignition of discharge
	\itemPar{gas} helium nitrogen mixture
	\itemPar{flow} helium $100\,\text{sccm}$
	\item nitrogen $0.05\,\text{sccm}$
	\itemPar{pressure} $p=1000\,\text{mbar}$
	\itemPar{frequency} $f=5\,\text{kHz}$
	\itemPar{shape} square wave
	\itemPar{amplitude} $\hat U_\text{appl}=1200\,\text{V}$
	\leer
	\itemwas measurement of Stark splitting
	\itemPar{lens} $h_\text{lens}=7.2:0.05:5.8\,\text{in}$
	\itemPar{hor. slit} $s_\text{hor}=0.2\,\text{mm}$
	\itemPar{filter} OG1 filter
 	\itemPar{MC} $s_\text{MC,in}=1\,\text{mm}$
	\item $s_\text{MC,out}=1\,\text{mm}$
	\item $g_\text{MC}=1800\,\text{mm}^{-1}$ at $500\,\text{nm}$
	\item $\lambda_\text{MC}=[587.65,667.98,690.0,706.66,728.31]\,\text{nm}$
	\leer
	\itemPar{R\&S RTO 1024} sample Rate: $500\,\text{MSa}/\text{s}$
	\item resolution: $2\,\text{ns}$
	\item record length: $2\,\text{kSa}$
	\item acquisition time: $4\,\text{\textmu s}$
	\item CH1: applied voltage, $200\,\text{V}/\text{DIV}$
	\item CH2: total charge, $4\,\text{V}/\text{DIV}$
	\item CH3: PMT signal, $40\,\text{mV}/\text{DIV}$, $-160\,\text{mV}$ offset
	\item CH4: Rogowski coil, $100\,\text{mV}/\text{DIV}$
	\item trigger: CH4 at $-150\,\text{mV}$, falling slope
	\item $15000$ averages
	\itemPar{PC} 5 total wavelength scans at each position
	\leer
	\itemPar{Files} 16001:16725
	\leer
	\itemwas measurement of full period
	\itemPar{Files} 17001
\end{Liste}

\neuertag{2016}{06}{20}{line intensity ratio}
\begin{Liste}
	\itemwas ignition of discharge
	\itemPar{gas} pure helium
	\itemPar{flow} helium $100\,\text{sccm}$
	\itemPar{pressure} $p=1000\,\text{mbar}$
	\itemPar{frequency} $f=5\,\text{kHz}$
	\itemPar{shape} sine wave
	\itemPar{amplitude} $\hat U_\text{appl}=1200\,\text{V}$
	\leer
	\itemwas line intensity ratio
	\itemPar{lens} $h_\text{lens}=5.8:0.05:7.2\,\text{in}$
	\itemPar{hor. slit} $s_\text{hor}=0.2\,\text{mm}$
	\itemPar{filter} OG1 filter
 	\itemPar{MC} $s_\text{MC,in}=1\,\text{mm}$
	\item $s_\text{MC,out}=1\,\text{mm}$
	\item $g_\text{MC}=1800\,\text{mm}^{-1}$ at $500\,\text{nm}$
	\item $\lambda_\text{MC}=[587.65,667.98,690.0,706.66,728.31]\,\text{nm}$
	\leer
	\itemPar{R\&S RTO 1024} sample Rate: $100\,\text{MSa}/\text{s}$
	\item resolution: $10\,\text{ns}$
	\item record length: $2\,\text{kSa}$
	\item acquisition time: $20\,\text{\textmu s}$
	\item temporal window: from $-6\,\text{\textmu s}$ to $14\,\text{\textmu s}$
	\item CH1: applied voltage, $200\,\text{V}/\text{DIV}$
	\item CH2: total charge, $2\,\text{V}/\text{DIV}$
	\item CH3: PMT signal, $40\,\text{mV}/\text{DIV}$, $-160\,\text{mV}$ offset
	\item CH4: Rogowski coil, $40\,\text{mV}/\text{DIV}$
	\item trigger: CH2 at $\-3.5,\text{V}$, rising slope
	\item $28000$ averages
	\itemPar{PC} 10 total wavelength scans at each position
	\leer
	\itemPar{Files} 30000:31450
\end{Liste}

\neuertag{2016}{06}{22}{Stark spectroscopy}
\begin{Liste}
	\itemwas ignition of discharge
	\itemPar{gas} helium with small admixture of nitrogen
	\itemPar{flow} helium $100\,\text{sccm}$
	\item nitrogen $0.05\,\text{sccm}$
	\itemPar{pressure} $p=1000\,\text{mbar}$
	\itemPar{frequency} $f=5\,\text{kHz}$
	\itemPar{shape} square wave
	\itemPar{amplitude} $\hat U_\text{appl}=1200\,\text{V}$
	\leer
	\itemwas Stark spectroscopy
	\itemPar{lens} $h_\text{lens}=6.8:0.1:7.1\,\text{in}$
	\itemPar{hor. slit} $s_\text{hor}=0.2\,\text{mm}$
	\itemPar{filter} polarization filter
 	\itemPar{MC} $s_\text{MC,in}=0.1\,\text{mm}$
	\item $s_\text{MC,out}=0.1\,\text{mm}$
	\item $g_\text{MC}=2400\,\text{mm}^{-1}$ at $500\,\text{nm}$
	\item $\lambda_\text{MC}=491.8:0.02:492.5\,\text{nm}$
	\leer
	\itemPar{R\&S RTO 1024} sample Rate: $500\,\text{MSa}/\text{s}$
	\item resolution: $2\,\text{ns}$
	\item record length: $1\,\text{kSa}$
	\item acquisition time: $2\,\text{\textmu s}$
	\item temporal window: from $-0.6\,\text{\textmu s}$ to $1.4\,\text{\textmu s}$
	\item CH1: applied voltage, $200\,\text{V}/\text{DIV}$
	\item CH2: total charge, $4\,\text{V}/\text{DIV}$
	\item CH3: PMT signal, $40\,\text{mV}/\text{DIV}$, $-160\,\text{mV}$ offset
	\item CH4: Rogowski coil, $100\,\text{mV}/\text{DIV}$
	\item trigger: CH2 at $\-150\,\text{mV}$, falling slope
	\item $50000$ averages
	\itemPar{PC} 10 wavelength scans (inner loop) at each position
	\leer
	\itemPar{Files} 35000:36440
\end{Liste}

%\input{Literatur}
\cleardoublepage
%\bibliography{../Literatur/Literaturdatenbank}
%gerplain, gerunsrt, geralpha, gerabbrv
% vordefiniert: plain, unsrt, alpha, abbrv
%\bibliographystyle{geralpha}
\bibliographystyle{alpha}


\end{document}